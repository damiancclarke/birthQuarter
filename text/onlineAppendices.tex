\documentclass[a4paper, 12 pt]{article}

\usepackage{amsfonts}
\usepackage{amsmath}
\usepackage{amsthm}
\usepackage{appendix}
\usepackage{bm}
\usepackage{booktabs}
\usepackage[usenames, dvipsnames]{color}
\usepackage{graphicx}
\usepackage{epstopdf}
\epstopdfsetup{update}
\usepackage{helvet}
\usepackage{hyperref}
\usepackage{indentfirst}
\usepackage{lscape}
\usepackage{morefloats}
\usepackage{natbib} \bibliographystyle{ecta}
\usepackage{setspace}
\usepackage{subcaption}
\usepackage[capposition=top]{floatrow}
\usepackage{subfloat}
\usepackage[latin1]{inputenc}
\usepackage{tikz}


\usetikzlibrary{trees}
\usetikzlibrary{decorations.markings}


\theoremstyle{plain}
\newtheorem{thm}{Theorem}
\newtheorem{cor}{Corollary}
\newtheorem{lem}[thm]{Lemma}
\newtheorem{proposition}{Proposition}
\newtheorem{assumption}{Assumption}
\newtheorem{definition}{Definition}

%MARGINS
 \topmargin   =  0.0in
 \headheight  =  -0.3in
 \headsep     =  0.7in
 \oddsidemargin= 0.0in
 \evensidemargin=0.0in
 \textheight  =  9.0in
 \textwidth   =  6.45in
%\setlength{\parindent}{4em}
%\setlength{\parskip}{1em}

\newcommand{\fmt}{.eps}
%\newcommand{\fmt}{.png}
\hypersetup{
    colorlinks=true,
    linkcolor=BlueViolet,
    citecolor=BlueViolet,
    filecolor=BlueViolet,
    urlcolor=BlueViolet
}


\renewcommand\figurename{Appendix Figure}
\renewcommand\tablename{Appendix Table}
\renewcommand\thesection{\Alph{section}}
\renewcommand*{\thepage}{A\arabic{page}}



\setcounter{page}{0}
\begin{document}
\begin{spacing}{1.4}
\begin{center}
\textbf{ONLINE APPENDICES} \\
\vspace{4mm}
From the paper: \\
\vspace{6mm}
{\large \textsc{Choosing Season of Birth:
The Role of Biological and Economic Constraints}} \\
Damian Clarke, Sonia Oreffice and Climent Quintana-Domeque
\end{center}

\tableofcontents


\setlength\parindent{0.25in}
\setlength\parskip{0.25in}




\newpage
\section{Replicating Results with Married and Unmarried}
\subsection{Tables}
\input{./../tables/tablesnvssall.tex}
\newpage
\subsection{Figures}
\input{./figuresnvssall.tex}

\newpage
\section{Additional Heterogeneity Tests (USA)}
\subsection{Results for Black Mothers Only}
\begin{landscape}
  \input{../results/nvss/regressions/NVSSBinaryBlack.tex}
\end{landscape}

\subsection{Results Seperated by Birth Gender}
\begin{figure}[htpb!]
  \begin{center}
    \caption{Minimum Monthly Temperature in the State and Birth Frequency (Females)}
    \label{bqFig:coldTeach}
    \begin{subfigure}{.5\textwidth}
      \centering
      \includegraphics[scale=0.55]{./../results/nvss/graphs/youngTempCold_female.eps}
      \caption{Young Mothers (28-31)}
      \label{fig:Educ}
    \end{subfigure}%
    \begin{subfigure}{.5\textwidth}
      \centering
      \includegraphics[scale=0.55]{./../results/nvss/graphs/oldTempCold_female.eps}
      \caption{Older Mothers (40-45)}
      \label{fig:NonEduc}
    \end{subfigure}
  \end{center}
  \floatfoot{\textsc{Notes to figure}: State averages of good season are plotted against the
    coldest average monthly temperature in the state.  The sample consists of only births where
    the child is female.}
\end{figure}

\begin{figure}[htpb!]
  \begin{center}
    \caption{Minimum Monthly Temperature in the State and Birth Frequency (Males)}
    \label{bqFig:coldTeach}
    \begin{subfigure}{.5\textwidth}
      \centering
      \includegraphics[scale=0.55]{./../results/nvss/graphs/youngTempCold_male.eps}
      \caption{Young Mothers (28.31)}
      \label{fig:Educ}
    \end{subfigure}%
    \begin{subfigure}{.5\textwidth}
      \centering
      \includegraphics[scale=0.55]{./../results/nvss/graphs/oldTempCold_male.eps}
      \caption{Older Mothers (40-45)}
      \label{fig:NonEduc}
    \end{subfigure}
  \end{center}
  \floatfoot{\textsc{Notes to figure}: State averages of good season are plotted against the
    coldest average monthly temperature in the state. The sample consists of only births where
    the child is male.}
\end{figure}

\begin{landscape}
  \input{../results/nvss/regressions/NVSSBinarygirls.tex}
\end{landscape}
\begin{landscape}
  \input{../results/nvss/regressions/NVSSBinaryboys.tex}
\end{landscape}



\clearpage
\section{Replicating Results with Spanish Birth Data}
\subsection{Tables}
\input{./../tables/SpainTables.tex}
\clearpage
\subsection{Figures}
\input{./spainFigures.tex}
\clearpage


\section{Teacher Temperature Results}
\begin{figure}[htpb!]
  \begin{center}
    \caption{Minimum Monthly Temperature in the State and Birth Frequency (All)}
    \label{bqFig:coldTeach}
    \begin{subfigure}{.5\textwidth}
      \centering
      \includegraphics[scale=0.55]{./../results/ipums/graphs/teachersTempCold.eps}
      \caption{Education Workers}
      \label{fig:Educ}
    \end{subfigure}%
    \begin{subfigure}{.5\textwidth}
      \centering
      \includegraphics[scale=0.55]{./../results/ipums/graphs/nonteachersTempCold.eps}
      \caption{Non-Education Workers}
      \label{fig:NonEduc}
    \end{subfigure}
  \end{center}
  \floatfoot{\textsc{Notes to figure}: State averages of good season are plotted against the
    coldest average monthly temperature in the state. Panel A includes all workers who are in
    ``Education, Training and Library Occupations'', while Panel B includes all other workers.
  }
\end{figure}

\begin{figure}[htpb!]
  \begin{center}
    \caption{Minimum Monthly Temperature in the State and Birth Frequency (28-31)}
    \label{bqFig:coldTeach}
    \begin{subfigure}{.5\textwidth}
      \centering
      \includegraphics[scale=0.55]{./../results/ipums/graphs/teachersTempCold_2831.eps}
      \caption{Education Workers}
      \label{fig:Educ}
    \end{subfigure}%
    \begin{subfigure}{.5\textwidth}
      \centering
      \includegraphics[scale=0.55]{./../results/ipums/graphs/nonteachersTempCold_2831.eps}
      \caption{Non-Education Workers}
      \label{fig:NonEduc}
    \end{subfigure}
  \end{center}
  \floatfoot{\textsc{Notes to figure}: State averages of good season are plotted against the
    coldest average monthly temperature in the state.  Only women aged 28-31 are included in the
    sample. Panel A includes all workers who are in ``Education, Training and Library
    Occupations'', while Panel B includes all other workers.
  }
\end{figure}


\begin{landscape}
  \input{../results/ipums/regressions/IPUMSTeachersCold.tex}
\end{landscape}







\bibliography{./refs}








\end{spacing}
\end{document}
