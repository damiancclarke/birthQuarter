\documentclass[a4paper, 12 pt]{article}

\usepackage{amsfonts}
\usepackage{amsmath}
\usepackage{amsthm}
\usepackage{appendix}
\usepackage{bm}
\usepackage{booktabs}
\usepackage[usenames, dvipsnames]{color}
\usepackage{graphicx}
\usepackage{epstopdf}
\epstopdfsetup{update}
\usepackage{helvet}
\usepackage{hyperref}
\usepackage{indentfirst}
\usepackage{lscape}
\usepackage{morefloats}
\usepackage{natbib} \bibliographystyle{ecta}
%\bibliographystyle{abbrvnat}\bibpunct{(}{)}{;}{a}{,}{,}
\usepackage{setspace}
\usepackage{subcaption}
\usepackage[capposition=top]{floatrow}
\usepackage{subfloat}
\usepackage[latin1]{inputenc}
\usepackage{tikz}
%\usepackage[pdf]{pstricks}

\usetikzlibrary{trees}
\usetikzlibrary{decorations.markings}


\theoremstyle{plain}
\newtheorem{thm}{Theorem}
\newtheorem{cor}{Corollary}
\newtheorem{lem}[thm]{Lemma}
\newtheorem{proposition}{Proposition}
\newtheorem{assumption}{Assumption}
\newtheorem{definition}{Definition}

%MARGINS
 \topmargin   =  0.0in
 \headheight  =  -0.3in
 \headsep     =  0.7in
 \oddsidemargin= 0.0in
 \evensidemargin=0.0in
 \textheight  =  9.0in
 \textwidth   =  6.45in
% \setlength{\parindent}{4em}
 \setlength{\parskip}{1em}

\newcommand{\fmt}{.eps}
%\newcommand{\fmt}{.png}
\hypersetup{
    colorlinks=true,
    linkcolor=BlueViolet,
    citecolor=BlueViolet,
    filecolor=BlueViolet,
    urlcolor=BlueViolet
}




\setcounter{page}{0}
\begin{document}



\title{\Large{\textsc{Choosing Season of Birth:}}\\ \Large{\textsc{The Role of Biological and Economic Constraints}}\thanks{\scriptsize{We thank seminar participants at University of Surrey, participants in the ``Family Economics'' Workshop (Barcelona Graduate School of Economics) June 2015, The Alicante Health Economics Workshop, and the Vienne LHEDC Workshop for helpful comments and suggestions. The usual disclaimers apply. Any errors contained in the paper are our own.}}}
\author{\small{Damian Clarke} \\ \small{University of Oxford} \and \small{Sonia Oreffice} \\ \small{University of Surrey \& IZA}  \and \small{Climent Quintana-Domeque} \\ \small{University of Oxford \& IZA}}

\date{\small{January, 2016} \\ \vspace{2mm} \small{\textbf{Preliminary Draft}}}


%\author{Damian Clarke  \and Sonia Oreffice  \and Climent Quintana-Domeque  }
%\author{Damian Clarke \\ University of Oxford \and Sonia Oreffice \\ University of Surrey and IZA \and Climent Quintana-Domeque \\University of Oxford and IZA}


\maketitle
\thispagestyle{empty}

\begin{abstract}
We provide novel estimates of women's decision of when to have their first birth in terms of fertility timing (young vs. old) and season of birth (which quarter), for non-Hispanic white women aged 25-45 in the US in 2005-2013. The prevalence of good season (quarters 2 and 3) is very significantly related to mother's age, as well as to her education and marital status, while those who do not undergo assisted reproductive technology procedures to achieve their first birth exhibit a much higher prevalence of good season births. The frequency of good season is also higher and more strongly related to mother's age in states where cold weather is more severe, reinforcing the interpretation that season of birth is a choice outcome. Finally, we find important interactions between good season of birth and a woman's labour market choices.  All of these facts point firmly to the fact that the season in which a baby is born is a choice variable made by the mother or family, potentially explaining a considerable portion of the quality difference observed in ``good season'' births. 
\end{abstract}
\emph{JEL Classification Codes}: I10, J01, J13.\\
\emph{Keywords}: quarter of birth, fertility timing, birth outcomes, wages.


\newpage
%\begin{spacing}{1.4}
\begin{doublespace}

%-------------------------------------------------------------------------------
\section{Introduction}
\paragraph{Motivation.} While the relevance of season of birth has been acknowledged at least since \citeauthor{Huntington38}'s \citeyear{Huntington38} book ``Season of Birth: Its Relation to Human Abilities'', it was not until the seminal article by \citet{AK1991}---in which quarter of birth was shown to be related to education and earnings in the USA---that season of birth became popular in economic research. Recent work has unveiled a variety of channels, beyond school cutoff laws, through which season of birth may affect adult outcomes, for example, its potential effects on birth outcomes. Indeed, a clear and consistent pattern of ``good'' and ``bad'' seasons has emerged. In the US, winter months are associated with lower birth weight, education and earnings, while Spring and Summer are found to be ``good'' seasons (e.g., \citealp{BucklesHungerman2013}; \citealp{CS2013}). However, as yet, no study has considered season of birth as a choice outcome.  In this paper we examine this premise.  We consider if season of birth is itself a choice variable in a mother's or families' childbearing decision, and quantify the trade-offs that mothers face (if any) when choosing season of birth.

\paragraph{Stylized facts.} We begin by documenting a set of rich and consistent stylised facts which suggest that season of birth is indeed an individual choice variable.  A significant ``good season'' gradient is observed by mother's age, by geographic variation in the climatic harshness of winter, and important interactions with women's education and labour market choices are observed.  These relationships hold conditional on competing explanations such as non-random fetal death and non-random gestational length by age, geographic location, and educational attainment, and hold in varying contexts in the Southern and Northern Hemisphere.  We turn to the discussion of these stylised facts in the body of the paper.

Given the prominence of fertility planning in balancing people's work and family life as well as the above stylized facts, it is hard to believe that season of birth may simply be a matter of chance. In addition, far from assuming that the average woman is aware that birth weight and the child's future earnings are affected by birth timing, it is sufficient to consider that the average woman has a sense that, on the one hand, winter months may be tougher birth months because of cold weather and higher disease prevalence,\footnote{According to the \citet{CDC2014}, from 1982-83 through 2013-14, the ``peak month of flu activity'' (the month with the highest percentage of respiratory specimens testing positive for influenza virus infection), has been February (14 seasons), followed by December (6 seasons) and January and March (5 seasons each): \href{http://www.cdc.gov/flu/about/season/flu-season.htm}{http://www.cdc.gov/flu/about/season/flu-season.htm}} and on the other, work commitments make it much easier to take time off with a Spring-Summer birth.\footnote{The report on fertility, family Planning, and women's health (\citealp{CDC1997}) notes that some women do not take maternity leave due to the timing birth relative to their job schedules, and gives the example of school teachers who deliver during summer break.}
%We examine these motivations and women's \emph{stated choices} in a complementary survey to support the extensive set of stylised facts observed in birth and labour market data.

\paragraph{This paper.} We first present novel correlates of season of birth, investigating women's decision of when to have their first birth in terms of fertility timing (young versus old) and season of birth (which quarter), for non-Hispanic white women aged 25-45. Using US Vital Statistics data from 2005 to 2013 on all first singleton births, we show that the prevalence of good season (quarters 2 and 3) is very significantly related to mother's age, as well as to her education and marital status. In addition, we find that women who do not undergo assisted reproductive technology (ART) are 3 percentage points more likely to give birth in the good season. This finding, which is robust to controlling for gestation length fixed effects, is consistent with season of birth being a choice outcome, if undergoing ART is associated to no longer being under control of birth timing. Moreover, if women undergoing ART cannot choose season of birth, we should expect to find no seasonality gap, and we present supportive evidence for this prediction. We then examine how birth outcomes, such as birth weight, prematurity ($<37$ weeks of gestation) and APGAR scores, are related to season of birth controlling for mother's characteristics. We find that being born in the good season is positively associated to better birth outcomes.

We then examine data on all births occurring in the American Community Survey (ACS) over a similar period.  Along with season of birth and mother's characteristics, ACS data allows us to examine the interaction of a child's season of birth with his mother's labour market decisions.  In the ACS we find, similarly, that young mothersand mothers in areas with harsher winter climates are considerably more likely to be observed giving birth in good seasons.  However, we also find important interactions with labour market choices.  We find that professions in which strong seasonality of work hours exist (such as educators), mothers are additionally more likely to choose good season of birth, and this holds conditional on observed age, education, and weather patterns.

Finally, we document that this result is not unique to the United States and its particular institutional and labour market context.  Using vital statistics from Spain we find that similar gradients in good season exist by age of mothers, by temperature, and by job type.  Once again, these findings are consistent with choice, as we can rule out that it is driven by selective fetal death or selective birth prematurity.

%In the second part of our investigation, we conduct our own survey to elicit information on the choices which women make (or are believed to make), and the degree to which beliefs about the benefits of good season affect those choices.  We recruit a pool of online survey respondents from Amazon's Mechanical Turk\footnote{Mechanical Turk (or MTurk) is a service provided by Amazon which allows for the recrutiment of a pool of workers to respond to a survey or perform other online tasks for payment.  The use of MTurk in experimental research is growing.  For further discussion, see for example \citet{Berinskyetal2012}.}, and ask respondents about their beliefs and choices.  This survey is currently being implemented, and will provide additional information to complement observational results from the US and Spanish vital statistical results, and the US survey data described above.


\paragraph{Related literature.} While \citet{CS2013} explain the first quarter of birth disadvantage through the negative impact of the disease environment on birth weight and gestational weeks in cold months, \citet{BucklesHungerman2013} emphasize the role of maternal characteristics in shaping the later socioeconomic disadvantage of winter-born individuals, showing that the mothers of those children are significantly less educated, less likely to be married or white, and more likely to be teenagers.\footnote{\citet{AlbaCaceres2014} describe similar findings for Chile and Spain.} Recent work by \citet{Barrecaetal2015} suggests that individuals may make short shifts in conception month in response to very hot days, with resulting declines and rebounds in following months.  Apart from the work by \citet{BucklesHungerman2013}, which may suggest the possibility that season of birth is \emph{not} random, and \citet{Barrecaetal2015}'s work on temperature and short-term shifts, there is a literature on ``exact'' birth timing analyzing the joint decision of parents and physicians to alter the delivery of an already existing pregnancy (in response to non-medical incentives). \citet{Shigeoka2015}, focusing on the distribution of births between December and January,  finds that in Japan many births are shifted one week forward around the school entry cutoff date. \citet{DCChandra1999} and \citet{LaLumiaetal2015} report that in the US parents may move expected January births backwards to December to gain tax benefits, while in Australia \citet{GansLeigh2009} estimate that parents moved forward June deliveries to become eligible for a newly introduced ``baby bonus''. Fewer births are documented on holidays \citep{Rindfuss1979} and weekends \citep{Gould2003}, less auspicious dates \citep{Almond15} or on medical professional meeting dates \citep{GLV2007}. Although this body of evidence clearly shows that parents may be willing and able to manipulate birth timing, it represents a choice made well after conception occurs. To the best of our knowledge, ours is the first economic analysis of the planning of season of birth.

\paragraph{Structure of the paper.} Section \ref{scn:data} describes the data sources. Section \ref{scn:results} presents the reduced-form estimates. Section 4 develops a structural model and provides estimates of the ``value'' of good season of birth. Section 5 concludes the paper.

%-------------------------------------------------------------------------------
\section{Data Sources and Descriptive Statistics}
\label{scn:data}
\subsection{Birth Data}
\label{bqSscn:USAdata}
Data on all births occurring each year in the United States are collected from birth certificate records, and publicly released as the National Vital Statistics System (NVSS) by the National Center of Health Statistics. These data are available for download for all years (inclusive) between 1968 and 2013, with all registered births in all states and the District of Columbia reported from 1984 onwards.\footnote{Prior to 1984, a 50\% sample was released for those states which did not submit their birth records on electronic, machine readable tape \citep{Martinetal2015}.}  In total, more than 99\% of births occurring in the country are registered \citep{Martinetal2015}. Our main estimation sample consists of birth years 2005-2013, and we retain all first births to white, non-Hispanic mothers.  Table \ref{bqTab:SumStatsNVSS} provides descriptive statistics of all births occurring to this group of mothers, and the principal covariates used in our analysis.  Restricting to non-hispanic white mothers who have their first birth results in 8,339,931 records for live births, including twins but excluding triplets and above. %In our main analysis, we restrict our sample to singletons, so that the estimation sample consists of 4,711,449 first births.

The birth certificate data record important information on births and their parents (mostly on mothers). For the mother, this includes age, marital status, education, smoking status during pregnancy, and assisted reproductive technology (ART) use. For the newborn, in addition to place and time of birth, measures include gestation (in weeks), birth weight, and one- and five-minute APGAR scores. However, birth certificates have gone through two important revisions in the variables reported:\ one in 1989 and the other in 2003.  These revisions (described fully in \citealp{NCHS2000}) were implemented by states at different points in time.  Prior to 2005, all states had fully incorporated the 1989 revision.  In the most recent wave of birth certificate data (2013), 41 states, containing 90.2\% of all births had switched to the more recent 2003 revision.  Importantly, the revised data include a different measure of education, a wider range of birth outcomes, and do not include the mother's smoking status. ART use information was first released in 2009. These changes mean that we do not have information for all variables over the whole period of analysis.\footnote{Complete details of missing variables are available in Table \ref{bqTab:SumStatsNVSS}, and further details regarding birth certificate revisions and the effect on reported variables and representativeness of the country as a whole are provided in appendix \ref{bqScn:datApp}.}  As such, our principal estimation sample is restricted only to those women for whom all covariates are recorded.  

Table \ref{bqTab:SumStatsMain} presents summary statistics just for our main estimation sample.  This consists of white, non-hispanic first-time mothers who are issued an updated birth certificate with education recorded. In the body of the paper we restrict our analysis to married women aged 25-45.\footnote{The analysis is replicated for all married and unmarried women in the online appendix to this paper.  Figure \ref{bqFig:NVSSages} displays the histogram of ages of first-time married mothers, and the proportion of these women who fall between 25-45 and hence are included in our main sample.}  This results in a sample of 2,260,745 births, 2,259,553 of which have gestation length recorded, and hence for whom conception month is known.

%Finally, from 2005
%onwards public use data do not contain geographic detail released in earlier
%waves of data.  %We applied to the National Association for Public Health
%Statistics and Information Systems (NAPHSIS) to receive state identifiers for all
%births for all time periods of our study.%\footnote{This data is freely available upon application and NAPHSIS review. Full details are provided on the web at \href{http://www.cdc.gov/nchs/nvss/dvs_data_release.htm} % {http://www.cdc.gov/nchs/nvss/dvs\_data\_release.htm}.}

%Summary statistics are provided in table \ref{bqTab:SumStatsNVSS}.  Of first-%
%time mothers in the USA, 84\% are married, and 97\% are aged under 40 years.
%The full distribution of mother's age at first birth for 25-45 year-olds is
%presented in figure \ref{bqFig:NVSSages}.  The modal age at first birth of this
%sample of women in the USA is 28 years, and the mean age is 30.24 years. A high
%proportion of white mothers from 2005-2013 report having at least some
%post-secondary education (87\%), where this category includes ``some college
%credit, but not a degree'', as well as associate or bachelor degrees and above.
%Of the subset of states and years in which smoking status is reported, 6\% of
%women report smoking at some point during pregnancy.  Of all births, slightly
%more children are born in quarters 2 or 3 (``good season''), at 51\%.  In terms
%of birth outcomes, 7\% of babies are classified as low birth weight (LBW) given
%that they weigh less than 2,500 grams at birth (the average birth weight is
%slightly more than 3,300g), 10\% are born premature (less than 37 weeks of
%gestation), and 3\% of recorded births are twins.

%-------------------------------------------------------------------------------
%\subsection{Spain Birth Data}
%We augment results from USA using data on births from Spain.

%-------------------------------------------------------------------------------
\subsection{Mother's Occupation and ACS Data}
In order to supplement analysis using full vital statistics data described above,
we use the American Community Survey (ACS) conducted by the United States Census
Bureau.  The ACS is a mandatory survey conducted on a representative 1\% of the
US population every year.  Along with details on births to all women, we observe
their labour market status (occupation), which is classed using the same
occupation codes as in recent census data.  For the analysis using ACS data, we
use surveys from each of the years analysed above, namely 2005-2014.

We extract all women who are household heads, spouses, or unmarried partners and
who have given birth in the year of the survey.  This results in the sample of
all children contained in the ACS data who were born in the period 2005-2014. We
then keep data on all women who correspond to our main estimation sample: 25-45
year old white, non-hispanics, who have had their first birth in the year in
question.  Given that we are interested in labour market outcomes, for this
analysis we retain only women who were employed in the last 5 years in non-%
military occupations that have a sample of at least 500 women over the entire
range of survey years.

%-------------------------------------------------------------------------------
\subsection{Weather and Unemployment Data}
A number of other data sources are consulted, and merged with birth data to provide time-varying coverage of local conditions at the time of conception, including measures of weather and unemployment. These are calculated at the year by month and state level, and are merged by conception (not birth) month.  We are able to calculate both conception and birth month, given that gestation is reported in the birth data.

Temperature data are provided by the National Centers for
Environmental Information from 1895 onwards, updated monthly.  We collate
measures of monthly means, maxima and minima for each state, year and month
over our time period of analysis, as described in \citet{Voseetal2014}. These are
available for all states with the exception of Hawaii and the District of Columbia (DC). We assign births that take place in DC temperature data from Maryland, a contiguous state. Unemployment data at the level of the state, year and month is
created from the Bureau of Labor Statistics' (BLS) online monthly time series
data.\footnote{Full records are available at
\href{http://download.bls.gov/pub/time.series/la}%
{http://download.bls.gov/pub/time.series/la}.} These data come from the Local
Area Unemployment Statistics (LAUS) Series, and are available for all states
plus DC for the entire time period of interest.

%-------------------------------------------------------------------------------
\subsection{Descriptive Statistics}

Summary statistics for married, white, non-Hispanic mothers aged 25-45 and their children, where the unit of observation is the first birth, are presented in Table \ref{bqTab:SumStatsMain}. The first panel of the table shows that women are on average 30 years old, and 97\% are aged below 40 by the time of their first birth (``younger''). Figure \ref{bqFig:NVSSages} displays the absolute frequencies of first births by mother's age for married (biological) mothers and our sample (25-45). While essentially there are no first births to women above 45, women younger than 25 represent about 20\% of first births to married women.  For all women (including unmarried), the number of first time mothers is much higher, with a substantial percentage of them being teenage pregnancies (see online appendices for further details). For those birth certificates with available mother's education information, 77\% have at least some college education; for those with non-missing smoking information, 3\% reported having smoked during pregnancy. Finally, for the five most recent years in our sample (2009-2013), we have information on the use of ART procedures: 1\% of the women report having used them to achieve their first birth.

In the second panel, we present detailed information on birth outcomes. 52\% of babies to first-time, married mothers are born in the good season, defined as quarters 2 and 3; taking into account gestational length, a similar proportion (52\%) of the newborns were planned for the good season. It is noteworthy that in the US none of the public holidays falls any close to the frontiers between the good and bad seasons defined above\footnote{Nationally Observed Public Holidays are: New Year's Day, Martin Luther King Jr. Day, Presidents' Day, Memorial Day, July 4, Labor Day, Columbus Day, Veteran's Day, Thanksgiving, Christmas Day.}. Regarding gender and multiple births, 49\% are girls while 2\% are twins (triplets and above where dropped from our sample); in our main analysis, we focus on singletons. Finally, we have information on birth ``quality'' measures, including birth weight, prematurity ($<37$ weeks of gestation) and APGAR score. The averages of these measures (3,353 grams, 8\%, 8.8, respectively) are consistent with those from previous studies.

We focus on first births, given that higher-order births also involve the additional decision of birth spacing and the role of experience, possibly underestimating the determinants of the choice of season of birth if planning improves with higher-order pregnancies. In the same vein, we consider only singleton births, although we use second-births and twin birth data in the sensitivity analysis, along with robustness checks on school entry rules, earlier years of data, and heterogeneity by socioeconomic status.

Table \ref{bqTab:singleSum} investigates the seasonality of birth by mother's age in binary age groups, and education (no college vs. at least some college). Panel A shows that young women (aged 28-31) are 4.5 percentage points (pp) more likely to give birth in the good season than in the bad season (52.2\% good vs. 47.8\% bad), whereas for older women the odds are virtually 50-50 (50.1\% good vs. 49.9\% bad). In Panel B we also observe that more educated women have a higher probability of giving birth in the good season (51.9\% vs. 48.1\%), while the gap for less educated women is 1.9 pp (50.9\% vs. 49.1\%).\footnote{The same type of investigation is developed with Spanish birth certificate data for the years 2007-2013 in a country with a much more generous maternity leave environment than the US. That is, this allows us to strengthen our interpretation of the choice nature of season of birth and to examine its relationship with mothers' labor force participation and occupation, information that is not at all recorded in the US certificates.}

Figure \ref{fig:goodByAge} highlights the seasonality gap by age group, as well as the justification of the particular definition of age groups in table \ref{bqTab:singleSum} and analysis in the remainder of the paper. This figure plots the frequency of good season for each age, compared to the omitted base group of 40-45 year olds.  Two features are worth mentioning. First, there is a decreasing gap in age from 25 to 45. In particular, the relative prevalence of good season is highest (more than 3 pp) for mothers aged 25-34, while it is essentially zero for mothers aged 40-45. Second, the relationship between seasonality gap and age is non-monotonic: The gap increases as women approach the age of 28, is approximately flat up until the age of 31, and then follows a downward trajectory for women aged 32-39. While the former feature is consistent with biological constraints whereby younger women have more flexibility to optimally time their births, the latter suggests that the prevalence of good season of birth cannot be entirely accounted for by the higher biological ability of young mothers to engage in optimal planning.


%-------------------------------------------------------------------------------
\section{Reduced-form estimates}
\label{scn:results}
\subsection{Births, Mother Characteristics, and Local Conditions}
  Figure \ref{bqFig:YoungvOld} shows the gap between the fraction of first births to ``younger'' (28-31) and ``older'' (40-45) women by month: The gap is positive in the months representing the ``good'' season (April to September) and negative in the ``bad'' season (October to March). This finding is consistent with ``younger'' mothers being less biologically constrained than ``older'' mothers when making their fertility decision, \emph{ceteris paribus}. Figure \ref{fig:mapYoung} reveals that, for ``younger'' women, good season is more prevalent in the North of the US than in the South. However, this pattern does not hold for ``older'' women, as we can see in Figure \ref{fig:mapOld}. Specifically, among ``younger'' women, a much higher proportion of good season births are observed in the northern states where the winter temperature is more severe. Interestingly, there is a North-South gradient, southern states with milder Winter exhibit lower proportion of births in good seasons. Strikingly, no such geographical pattern is observed of first births to ``older'' mothers, with the proportion of good season births appearing to be unrelated to geographic location of the state.

We further investigate whether the geographical differences in the prevalence of good season are due to weather conditions in Figure \ref{fig:tempUSA}. If women choose season of birth at all, they may be more willing to give birth in the Spring or Summer, at least in states with more severe cold weather in Winter. We plot the percentage of ``younger'' women giving birth in the good season against the coldest monthly average by state: The pattern is spectacular. There is a strong linear negative association between these two variables (correlation coefficient = $-0.668$). Interestingly, we do not find such a relationship for older women (correlation coefficient = $0.108$). Finally, in Figure \ref{bqFig:excessTemp}, we present additional evidence that the seasonality of births is strongly related to weather: The US (Northern hemisphere) seasonality patterns of birth are completely reversed in Chile (Southern hemisphere).


\paragraph{Season of birth correlates.} In Table \ref{testlabel} we investigate the relationship between good season of birth (quarters 2 and 3) and mother's age. In column 1 we see that ``younger'' women are approximately 2 pp more likely to have their first child in the good season than ``older'' women (aged 40-45). These age dummies reflect the graphical pattern observed in figure \ref{bqFig:YoungvOld} of a non-monotonic relationship between age and good season with a peak good season age of 28 years. This difference is robust to the addition of control variables (columns 2-4): state and year fixed effects, education (an indicator for having some college or above), and (an indicator for) smoking during pregnancy. In addition, high-educated women (or married women) are at most only 1 pp more likely to have their first born child in the good season than their counterparts, which is only half of the difference between younger and older women. Women who smoked during pregnancy are at most 1 pp less likely to have their first birth in the good season. Hence, a mother's age seems to be the most relevant driving force behind season of birth. Finally, in columns 5-7, we investigate the role of undergoing an ART procedure. Since this information is available only from 2009 to 2013, we replicate column 4 with this restricted sample in column 5, finding the same results. In column 6 we include an ART indicator (1 if the birth did not happen through an ART procedure, 0 otherwise), and estimate a strongly positive significant coefficient: Women who do not undergo ART are 3 pp more likely to give birth in the good season. This finding, which is robust to controlling for gestation length fixed effects, is consistent with season of birth being a choice variable, if undergoing ART is associated to no longer being under control of birth timing.% of your future child.


\paragraph{Placebo test: ART versus non-ART users.} If women undergoing ART cannot choose season of birth, we should expect to find no seasonality gap. However, women undergoing ART who are very young (20-24), well below the mean age of mother at first birth in the US (26 years in 2013; see \citealp{Martinetal2015}), are likely to suffer from serious infertility problems and may be those who end up in the bad season. These two features are precisely reported in Figure \ref{fig:birthDiffART}. Instead, Figure \ref{fig:DiffART} reports the pattern described above for the non-ART users.\footnote{See Figure \ref{fig:MonthNoART} for a month-by-month comparison in the seasonality gap by ART status.} Table \ref{tab:ART2024} summarizes these graphical findings in regression format.

While these results suggest that ART users are actually on average more likely to give birth during the ``bad season'', no systematic or statistically significant difference is observed when comparing older to younger women (figure \ref{fig:concepAbsART}).  Indeed, when examining the distribution of ART births over the year, the entire difference in the proportion of good season births appears to be driven by a large reduction of ART conceptions occurring in January.  This is in line with seasonality in opening hours of ART clinics, which in many cases have extended periods of closure in December.  This is supported by anecdotal evidence from an online search of clinic opening hours.

\paragraph{Birth outcomes correlates.} In Table \ref{tab:quality} we investigate the correlates of birth outcomes. Babies born in the good season tend have better outcomes at birth, after controlling for mother characteristics: They are 8 grams heavier; they are 0.1 pp less likely to be LBW ($<2,500$ grams); they have on average 0.02 more weeks of gestation; and they are 0.02 pp less likely to be premature ($<37$ weeks of gestation). Younger women tend to have babies with higher ``quality'' at birth: Their babies tend to be between 90-105g heavier; they are 4 pp less likely to be LBW; 0.7 pp less likely to be VLBW ($<1,500$ grams); they have on average 0.5 more weeks of gestation; they are 4 pp less likely to be premature; and they score 0.06 additional units in the APGAR score. In addition, high-educated (and married) women tend to have babies with better outcomes at birth. Finally, women who smoke in pregnancy have babies who are 171 grams lighter, consistent with the findings in Lien and Evans (2005), who use an instrumental variable approach and find that maternal smoking reduces mean birth weight by 182 grams.


\subsection{Robustness checks}
We examine a number of alternative specifications and samples to test the robustness of ``good season'' choice in varying contexts.  The inclusion of state specific linear trends and unemployment rate at season of conception leads to essentially no changes in estimated coefficients (table 5).  In appendix tables, we consider the additional sample of second births and run our main regressions of good season of birth on maternal characteristics, finding the same pattern of results and significance, with slightly larger estimated coefficients (Table 20). Interestingly, when we instead focus on the sample of twin first births we do not find any statistically significant association between the prevalence of good season among twins and maternal characteristics such as age or education (Table 21). Twins represent only 2\% of the total of first births, and nowadays mainly arise as an effect of non-ART and ART procedures in the presence of infertility problems: We believe that the above evidence is consistent with our interpretation of season of birth as a choice variable, since for women undergoing ART treatments birth timing and planning is often out of their control. Finally, including fetal deaths in our regressions does not alter our findings (Table 22).

To check the role of mother's age, we also run regressions using age (and age squared) as a continuous variable, or with an indicator of being 25-34 years old instead of 25-39 for the younger group. Table 23 provides evidence consistent with the relevant role of mother's age in determining season of birth, in line with the quadratic relationship between age and season of birth described in figure \ref{bqFig:YoungvOld}. Finally, when running the birth quality regressions on the sample of second births or twins, the estimates confirm our previous findings: good season is positively significantly associated to birth quality indicators for second births (table 25), but is considerably less so for twin first births (Tables 13).

In the online appendix to this paper, we replicate the entire analysis using Spanish vital statistical data.  Despite the considerably different context, both in terms of the months which are ``good season'' and the institutional context surrounding maternal leave policy, the results point to largely identical findings.  Once again we find that younger mothers are significantly more likely to give birth in good seasons, the prevalence of good season depends strongly on the climatic harshness of the place of birth, and the follow on effect of good season on birth quality outcomes is significant.  Full analysis is reproduced in online appendix C.


\subsection{Births and Maternal Occupation}
By using US Census data, we are able to replicate the above findings, while also observing the mother's stated observation.  This allows us not only to test the veracity of our results, but more importantly, to test whether ``good season'' choices of birth timing interact with a mother's labour market decisions.  There is considerable evidence that labour market flexibility effects women's job choices as well as partially explaining the pay gap \citep{Goldin2014}.  Here we test whether labour market flexibility and job type also interact directly with child bearing choices and timing.

Tables 9-13 replicate results from vital statistics data using the 1\% census sample contained in ACS data.  Despite being based on a much smaller sample, we see that all the principal conclusions of the birth certificate data are backed up in ACS results.  In table 14, we examine how good season choices interact with a mother's occupation class.  In column 2 we present results from a regression of good season on mother's age and two level occupation class from the census.  We see that, conditional on age and education, labour market decisions have an \emph{additional} impact on the likelihood of giving birth in quarters 2 or 3.  This is particularly striking among educators (``education, training and library), who have frequently have a long summer break which can be timed to align with child birth.  In figures 15-17 we find that these results hold even when conditioning on all of income, education and unemployment rates at conception.

In figure 18 we examine birth timing and occupation class by quarter.  As outlined above, educators are much more likely to time their births to aling with the beginning of their summer break (quarter 2), while other significant observations are more likely to target quarter 3: the most temperate birth quarter.  Indeed, in online appendix figures, the occupational advantages for teachers giving birth during vacations are found to be completely unrelated to weather.  Even in areas in which winters are mild, women working in educational occupations are found to prefer good season given the considerably longer labour market break that this offers after birth and before return to work.

%\section{Choice and the Biology--Economy Trade-off}

\section{Conclusion}
The effects of season of birth on newborn and adult socioeconomic outcomes have been widely documented across disciplines, where a clear and consistent pattern of ``good'' and ``bad'' seasons has emerged. This is the first analysis of season of birth as a choice that women may make, and to estimate the value of good season of birth in terms of birth weight and wages.

We document a consistent and clear series of stylised facts suggesting that women choose the seasonality of their birth.  Firstly, younger women, who have more remainging years for potential child bearing, are considerably more likely to time births to fall in the summer month.  Secondly, those women who engage in ART are found to be significantly less likely to give birth in good birth seasons, given that their ability to time births depends much more on the availability of IVF and other treatments at clinics.  Thirdly, we document that the probability of choosing good season of birth varies inversely with the plesantness of winter: when winters become harsher, the costs of a winter birth rise, and women are much more likely to time birth in summer. Finally, we find that birth timing decisions interact strongly with labour market choices.  Particularly, women who work in educational occupations are found to have a much larger proportion of births which fall in summer.

All in all, this points to birth timing within a year as a choice.  This finding has important implications on the question of why babies born in good seasons are healthier.  While conditions in utero \emph{are} more favourable, there is also a strong selection effect among young, educated, and employed mothers.

\newpage
%\bibliographystyle{./economet}
%\bibliography{./refs}
%\bibliographystyle{./jpe}

\bibliography{./refs}



\newpage
\section*{Tables}
\input{./../tables/sumStatsnvss.tex}
\input{./../tables/sumsinglenvss.tex}
\input{./../results/nvss/regressions/NVSSBinaryMain.tex}
\begin{landscape}
\input{./../tables/quarterHeterogeneity.tex}
\end{landscape}
\input{./../results/nvss/regressions/NVSSBinaryEdInteract.tex}
\input{./../results/nvss/regressions/NVSSBinaryYoung34.tex}
\input{./../results/nvss/regressions/NVSSseasonMLogit.tex}
\begin{landscape}
\begin{table}[htbp]\centering
\def\sym#1{\ifmmode^{#1}\else\(^{#1}\)\fi}
\caption{Birth Quality by Age and Season (NVSS 2005-2013)}
\scalebox{0.68}{
\begin{tabular}{l*{7}{c}}
\toprule
                    &\multicolumn{1}{c}{(1)}   &\multicolumn{1}{c}{(2)}   &\multicolumn{1}{c}{(3)}   &\multicolumn{1}{c}{(4)}   &\multicolumn{1}{c}{(5)}   &\multicolumn{1}{c}{(6)}   &\multicolumn{1}{c}{(7)}   \\
                    &       APGAR   & Birthweight   &   Gestation   &         LBW   &   Premature   &        Twin   &        VLBW   \\
\midrule
Aged 25-39          &       0.043***&     130.718***&       0.598***&      -0.055***&      -0.064***&      -0.054***&      -0.010***\\
                    &     [0.004]   &     [2.581]   &     [0.011]   &     [0.001]   &     [0.001]   &     [0.001]   &     [0.000]   \\
Bad Season          &       0.002   &      -5.653   &      -0.011   &       0.004** &       0.002   &       0.001   &      -0.001*  \\
                    &     [0.005]   &     [3.585]   &     [0.015]   &     [0.002]   &     [0.002]   &     [0.001]   &     [0.001]   \\
Young$\times$ Bad S &      -0.005   &      -4.781   &      -0.014   &      -0.001   &       0.001   &       0.001   &       0.002***\\
                    &     [0.005]   &     [3.638]   &     [0.015]   &     [0.002]   &     [0.002]   &     [0.001]   &     [0.001]   \\
College Educ        &       0.045***&      77.112***&       0.181***&      -0.025***&      -0.021***&       0.010***&      -0.006***\\
                    &     [0.001]   &     [0.917]   &     [0.004]   &     [0.000]   &     [0.000]   &     [0.000]   &     [0.000]   \\
&&&&&&&\\
Constant            &       8.756***&    3121.397***&      38.034***&       0.150***&       0.194***&       1.075***&       0.028***\\
                    &     [0.004]   &     [2.775]   &     [0.012]   &     [0.001]   &     [0.001]   &     [0.001]   &     [0.001]   \\
\midrule
R-squared           &        0.00   &        0.00   &        0.00   &        0.00   &        0.00   &        0.00   &        0.00   \\
Observations        &     3551931   &     3603294   &     3610749   &     3613920   &     3613920   &     3613920   &     3613920   \\
\bottomrule
\multicolumn{8}{p{15cm}}{\begin{footnotesize}Sample consists of all
first born children of US-born, white, non-hispanic mothers
\end{footnotesize}}\end{tabular}}\end{table}

\end{landscape}
\begin{landscape}
\input{./../tables/qualityHeterogeneity.tex}
\end{landscape}
\begin{landscape}
\input{./../results/nvss/regressions/QualityAllComb.tex}
\end{landscape}
\input{./../tables/sumStatsSpain.tex}
\input{./../tables/sumSpain.tex}
\begin{landscape}
\input{./../results/spain/regressions/spainBinary.tex}
\end{landscape}
\begin{landscape}
\input{./../results/spain/regressions/spainQualityEduc.tex}
\end{landscape}
\begin{landscape}
\input{./../results/spain/regressions/spainQualityGestFix.tex}
\end{landscape}

\input{./../tables/tablesipums.tex}

\newpage
\section*{Figures}
\begin{figure}[htpb!]
\centering
\caption{Differences in Prevalence of Good Season Of Birth}
\label{bqFig:YoungvOld}
  \centering
  \includegraphics[scale=0.82]{./../results/nvss/graphs/youngMonths.eps}
\floatfoot{\textsc{Notes to figure}: Each point and standard error comes from a regression
of conception month $x$ on a binary indicator of being young (28-31), versus older (40-45).
}
\end{figure}


\begin{figure}[htpb!]
\begin{center}
\caption{Differences in Prevalence of Good Season Of Birth by ART Usage}
\label{fig:MonthART}
\begin{subfigure}{.5\textwidth}
  \centering
  \includegraphics[scale=0.55]{./../results/nvss/graphs/youngMonthsART0.eps}
  \caption{No ART Usage}
  \label{fig:MonthNoART}
\end{subfigure}%
\begin{subfigure}{.5\textwidth}
  \centering
  \includegraphics[scale=0.55]{./../results/nvss/graphs/youngMonthsART1.eps}
  \caption{ART Usage}
  \label{fig:MonthYesART}
\end{subfigure}
\end{center}
\floatfoot{\textsc{Notes to figure}: Each point and standard error comes from a 
regression of birth month $x$ on a binary indicator of being young (28-31).  ART 
usage is only observed in birth data in the years 2009--2013.}
\end{figure}


\begin{figure}[htpb!] 
\begin{center}
  \centering  
  \caption{Difference in Good Season by Age Group}
  \includegraphics[scale=0.72]{./../results/nvss/graphs/goodSeasonAge.eps}  
  \label{fig:goodByAge} 
\end{center}
\vspace{-5mm}
\floatfoot{\textsc{Notes to figure \ref{fig:goodByAge}}: Coefficients and
standard errors are estimated by regressing ``good season'' on dummies
of maternal age.  Age groups 40-45 are omitted as the base group.  The full
sample consists of mothers aged 20-45.  For the omitted group, proportion 
good season (and standard error) is 0.497(0.001).}
\end{figure} 


\begin{figure}[htpb!]
\begin{center}
\caption{Age and Conception Month}
\label{bqFig:concepMonth}
\begin{subfigure}{.5\textwidth}
  \centering
  \includegraphics[scale=0.55]{./../results/nvss/graphs/conceptionMonth.eps}
  \caption{Proportion of Conceptions in Each Month}
  \label{fig:concepAbs}
\end{subfigure}%
\begin{subfigure}{.5\textwidth}
  \centering
  \includegraphics[scale=0.55]{./../results/nvss/graphs/conceptionMonthART.eps}
  \caption{Proportion of Conceptions (ART Only)}
  \label{fig:concepAbsART}
\end{subfigure}
\end{center}
\floatfoot{\textsc{Notes to figure \ref{bqFig:concepMonth}}: Month of conception
is calculated by subtracting the rounded number of gestation months (gestation in
weeks $\times$ 7/30.5) from month of birth.  Each line presents the proportion of
all births conceived in each month for the relevant age group.}
\end{figure}

\begin{figure}[htpb!]
\begin{center}
\caption{Education and Conception Month}
\label{bqFig:concepEduc}
\begin{subfigure}{.5\textwidth}
  \centering
  \includegraphics[scale=0.55]{./../results/nvss/graphs/conceptionMonthEducYoung.eps}
  \caption{28-31 Year Olds}
  \label{fig:educYoung}
\end{subfigure}%
\begin{subfigure}{.5\textwidth}
  \centering
  \includegraphics[scale=0.55]{./../results/nvss/graphs/conceptionMonthEducOld.eps}
  \caption{40-45 Year Olds}
  \label{fig:educOld}
\end{subfigure}
\end{center}
\floatfoot{\textsc{Notes to figure \ref{bqFig:concepEduc}}: Each line presents the 
proportion of all births conceived in each month for the relevant age group and
education level. Refer to figure \ref{bqFig:concepMonth} for additional notes.}
\end{figure}


\begin{figure}[htpb!]
\begin{center}
  \centering
  \caption{Good Season by State (Young)}
  \includegraphics[scale=0.7]{./../results/nvss/graphs/maps/youngGoodSeason.eps}
  \label{fig:mapYoung}
\end{center}
\end{figure}

\begin{figure}[htpb!]
\begin{center}
  \centering
  \caption{Good Season by State (Old)}
  \includegraphics[scale=0.7]{./../results/nvss/graphs/maps/oldGoodSeason.eps}
  \label{fig:mapOld}
\end{center}
\floatfoot{\textsc{Notes to figures \ref{fig:mapYoung}-\ref{fig:mapOld}}: 
State data consists of all first births to white non-hispanic women from 2005 
and 2013.  Figure \ref{fig:mapYoung} includes all mothers aged 28-31, while
figure \ref{fig:mapOld} includes mothers aged 40-45.}
\end{figure}


\begin{figure}[htpb!]
\begin{center}
\caption{Temperature and Good Quarter}
\label{fig:tempUSA}
\begin{subfigure}{.5\textwidth}
  \centering
  \includegraphics[scale=0.55]{./../results/nvss/graphs/youngTempCold.eps}
  \caption{Young Mothers}
  \label{fig:tempUSAYoung}
\end{subfigure}%
\begin{subfigure}{.5\textwidth}
  \centering
  \includegraphics[scale=0.55]{./../results/nvss/graphs/oldTempCold.eps}
  \caption{Old Mothers}
  \label{fig:tempUSAOld}
\end{subfigure}
\end{center}
\floatfoot{\textsc{Notes to figure \ref{fig:tempUSA}}: Each point 
represents a state average of the proportion of women giving birth in the 
good birth season between 2005 and 2013.  The dotted line is a fitted 
regression line.  Monthly temperature data is collected from the National 
Centers for Environmental Information.}
\end{figure}

\begin{figure}[htpb!]
\begin{center}
\caption{Birth per Month and Temperature: Various Countries}
\label{bqFig:excessTemp}
\begin{subfigure}{.5\textwidth}
  \centering
  \includegraphics[scale=0.55]{./../results/countries/excessMonthChileYoungTemp.eps}
  \caption{Chile}
  \label{fig:ChileTemp}
\end{subfigure}%
\begin{subfigure}{.5\textwidth}
  \centering
  \includegraphics[scale=0.55]{./../results/countries/excessMonthUSAYoungTemp.eps}
  \caption{USA}
  \label{fig:USATemp}
\end{subfigure}
\end{center}
\floatfoot{\textsc{Note to figure \ref{bqFig:excessTemp}}: 
Bars represent the difference between expected (evenly spaced) births and actual births.  
Dotted line represents average temperature in the whole of the country over the period
1990-2009 from the World Bank Climate Change Portal.  Births for USA are 2005-2013 and 
for Chile 2000-2012.}
\end{figure}
\vspace{5mm}

\begin{figure}[htpb!]
  \begin{center}
  \caption{Mother's Age at First Birth}
  \label{bqFig:NVSSages}
  \includegraphics[scale=0.82]{./../results/nvss/graphs/ageDescriptive.eps} 
  \end{center}
\end{figure}

\begin{figure}[htpb!]
  \begin{center}
  \caption{Difference in Births (\% Good Season - \% Bad Season)}
  \label{fig:NVSSbirthsAges}
  \includegraphics[scale=0.8]{./../results/nvss/graphs/birthQdiff_4Ages.eps}
  \end{center}
\end{figure}

\begin{figure}[htpb!] 
\begin{center}
  \centering  
  \caption{ART Conceptions by Month}
  \includegraphics[scale=0.7]{./../results/nvss/graphs/proportionMonthART.eps}  
  \label{fig:ARTMonth} 
\end{center}
\floatfoot{\textsc{Notes to figure \ref{fig:ARTMonth}}: Proportion of ART births
are calculated using data from 2012-2013 for our main sample.  The proportion
is calculated as: (ART conceptions)/(Non-ART Conceptions + ART Conceptions).}
\end{figure} 

\begin{figure}[htpb!]
\begin{center}
\caption{Difference in Births (\% Good Season - \% Bad Season)}
\label{fig:birthDiffART}
\begin{subfigure}{.5\textwidth}
  \centering
  \includegraphics[scale=0.55]{./../results/nvss/graphs/birthQdiff_4AgesART.eps}
  \caption{ART Usage}
  \label{fig:DiffNoART}
\end{subfigure}%
\begin{subfigure}{.5\textwidth}
  \centering
  \includegraphics[scale=0.55]{./../results/nvss/graphs/birthQdiff_4AgesNoART.eps}
  \caption{No ART Users}
  \label{fig:DiffART}
\end{subfigure}
\end{center}
\end{figure}

\begin{figure}[htpb!]
\begin{center}
\caption{Conceptions by Month (Arizona and Wisconsin)}
\label{fig:conceptionsStates}
\begin{subfigure}{.5\textwidth}
  \centering
  \includegraphics[scale=0.55]{./../results/nvss/graphs/conceptionMonthArizonaWisconsin_young.eps}
  \caption{Young Women (28-31)}
  \label{fig:conceptionsStatesYoung}
\end{subfigure}%
\begin{subfigure}{.5\textwidth}
  \centering
  \includegraphics[scale=0.55]{./../results/nvss/graphs/conceptionMonthArizonaWisconsin_old.eps}
  \caption{Old Women (40-45)}
  \label{fig:conceptionsStatesOld}
\end{subfigure}
\end{center}	
\floatfoot{\textsc{Note to figure \ref{fig:conceptionsStates}}: 
Sample consists of all first births from 2005-2013 to white, non-hispanic 25-45 year old 
mothers occurring in the states of Arizona (142,788 births) or Wisconsin (183,194 births).}
\end{figure}

\begin{figure}[htpb!]
  \begin{center}
  \caption{Prematurity by Mother's Age}
  \label{fig:ARTuse}
  \includegraphics[scale=0.8]{./../results/nvss/graphs/prematureAges.eps}
  \end{center}
\end{figure}


\begin{figure}[htpb!]
\centering
\caption{Differences in Prevalence of Good Season Of Birth}
\label{bqFig:YoungvOldIPUMS}
  \centering
  \includegraphics[scale=0.82]{./../results/ipums/graphs/youngQuarter.eps}
\floatfoot{\textsc{Notes to figure}: Each point and standard error comes from a regression
of conception month $x$ on a binary indicator of being young (28-31), versus older (40-45).
}
\end{figure}


\begin{figure}[htpb!] 
\begin{center}
  \centering  
  \caption{Difference in Good Season by Age Group}
  \includegraphics[scale=0.72]{./../results/ipums/graphs/goodSeasonAge.eps}  
  \label{fig:goodByAgeIPUMS} 
\end{center}
\vspace{-5mm}
\floatfoot{\textsc{Notes to figure \ref{fig:goodByAge}}: Coefficients and
standard errors are estimated by regressing ``good season'' on dummies
of maternal age.  Age groups 40-45 are omitted as the base group.  The full
sample consists of mothers aged 20-45.  For the omitted group, proportion 
good season (and standard error) is 0.506(0.008).}
\end{figure} 



\begin{figure}[htpb!]
\begin{center}
\caption{Age and Birth Quarter}
\label{bqFig:birthQuarterIPUMS}
%%\begin{subfigure}{.5\textwidth}
%%  \centering
  \includegraphics[scale=0.72]{./../results/ipums/graphs/birthQuarterAges.eps}
%%  \caption{Proportion of Conceptions in Each Month}
%%  \label{fig:concepAbs}
%%\end{subfigure}%
%%\begin{subfigure}{.5\textwidth}
%%  \centering
%%  \includegraphics[scale=0.55]{./../results/nvss/graphs/conceptionMonthART.eps}
%%  \caption{Proportion of Conceptions (ART Only)}
%%  \label{fig:concepAbsART}
%%\end{subfigure}
\end{center}
\floatfoot{\textsc{Notes to figure \ref{bqFig:birthQuarterIPUMS}}: Birth quarter 
is reported in ACS data. Each line presents the proportion of all births occurring
in each quarter for the relevant age group.}
\end{figure}


\begin{figure}[htpb!]
\begin{center}
\caption{Education and Birth Quarter}
\label{bqFig:concepEducIPUMS}
 \begin{subfigure}{.5\textwidth}
   \centering
   \includegraphics[scale=0.55]{./../results/ipums/graphs/birthQuarterEducYoung.eps}
   \caption{28-31 Year Olds}
   \label{fig:educYoungIPUMS}
 \end{subfigure}%
 \begin{subfigure}{.5\textwidth}
   \centering
   \includegraphics[scale=0.55]{./../results/ipums/graphs/birthQuarterEducOld.eps}
   \caption{40-45 Year Olds}
   \label{fig:educOldIPUMS}
 \end{subfigure}
 \end{center}
 \floatfoot{\textsc{Notes to figure \ref{bqFig:concepEducIPUMS}}: Each line 
presents the proportion of all births conceived in each month for the relevant 
age group and education level. Refer to figure \ref{bqFig:concepMonth} 
for additional notes.}
\end{figure}

 
 \begin{figure}[htpb!]
 \begin{center}
   \centering
   \caption{Good Season by State (Young)}
   \includegraphics[scale=0.7]{./../results/ipums/graphs/youngGoodSeason.eps}
   \label{fig:mapYoungIPUMS}
 \end{center}
 \end{figure}
 
 \begin{figure}[htpb!]
 \begin{center}x
   \centering
   \caption{Good Season by State (Old)}
   \includegraphics[scale=0.7]{./../results/ipums/graphs/oldGoodSeason.eps}
   \label{fig:mapOldIPUMS}
 \end{center}
 \floatfoot{\textsc{Notes to figures \ref{fig:mapYoungIPUMS}-\ref{fig:mapOldIPUMS}}: 
 State data consists of all first births to white non-hispanic women from 2005 
 and 2013.  Figure \ref{fig:mapYoungIPUMS} includes all mothers aged 28-31, while
 figure \ref{fig:mapOldIPUMS} includes mothers aged 40-45.}
 \end{figure}
 
 
 \begin{figure}[htpb!]
 \begin{center}
 \caption{Temperature and Good Quarter}
 \label{fig:tempUSAIPUMS}
 \begin{subfigure}{.5\textwidth}
   \centering
   \includegraphics[scale=0.55]{./../results/ipums/graphs/youngTempCold.eps}
   \caption{Young Mothers (28-31)}
   \label{fig:tempUSAYoungIPUMS}
 \end{subfigure}%
 \begin{subfigure}{.5\textwidth}
   \centering
  \includegraphics[scale=0.55]{./../results/ipums/graphs/oldTempCold.eps}
  \caption{Old Mothers (40-45)}
  \label{fig:tempUSAOldIPUMS}
\end{subfigure}
\end{center}
\floatfoot{\textsc{Notes to figure \ref{fig:tempUSAIPUMS}}: Each point 
represents a state average of the proportion of women giving birth in the 
good birth season between 2005 and 2014.  The dotted line is a fitted 
regression line.  Monthly temperature data is collected from the National 
Centers for Environmental Information.  States with less than 500 births
in ACS over the entire period of analysis are not displayed.}
\end{figure}

\begin{figure}[htpb!]
  \begin{center}
  \caption{Difference in Births (\% Good Season - \% Bad Season)}
  \label{fig:NVSSbirthsAgesIPUMS}
  \includegraphics[scale=0.8]{./../results/ipums/graphs/birthQdiff_4Ages.eps}
  \end{center}
\end{figure}
%%
%%\begin{figure}[htpb!] 
%%\begin{center}
%%  \centering  
%%  \caption{ART Conceptions by Month}
%%  \includegraphics[scale=0.7]{./../results/nvss/graphs/proportionMonthART.eps}  
%%  \label{fig:ARTMonth} 
%%\end{center}
%%\floatfoot{\textsc{Notes to figure \ref{fig:ARTMonth}}: Proportion of ART births
%%are calculated using data from 2012-2013 for our main sample.  The proportion
%%is calculated as: (ART conceptions)/(Non-ART Conceptions + ART Conceptions).}
%%\end{figure} 
%%
%%\begin{figure}[htpb!]
%%\begin{center}
%%\caption{Difference in Births (\% Good Season - \% Bad Season)}
%%\label{fig:birthDiffART}
%%\begin{subfigure}{.5\textwidth}
%%  \centering
%%  \includegraphics[scale=0.55]{./../results/nvss/graphs/birthQdiff_4AgesART.eps}
%%  \caption{ART Usage}
%%  \label{fig:DiffNoART}
%%\end{subfigure}%
%%\begin{subfigure}{.5\textwidth}
%%  \centering
%%  \includegraphics[scale=0.55]{./../results/nvss/graphs/birthQdiff_4AgesNoART.eps}
%%  \caption{No ART Users}
%%  \label{fig:DiffART}
%%\end{subfigure}
%%\end{center}
%%\end{figure}
%%
\begin{figure}[htpb!]
\begin{center}
\caption{Conceptions by Month (Arizona and Wisconsin)}
\label{fig:conceptionsStatesIPUMS}
\begin{subfigure}{.5\textwidth}
  \centering
  \includegraphics[scale=0.55]{./../results/ipums/graphs/birthQuarterArizonaWisconsin_young.eps}
  \caption{Young Women (28-31)}
  \label{fig:conceptionsStatesYoungIPUMS}
\end{subfigure}%
\begin{subfigure}{.5\textwidth}
  \centering
  \includegraphics[scale=0.55]{./../results/ipums/graphs/birthQuarterArizonaWisconsin_old.eps}
  \caption{Old Women (40-45)}
  \label{fig:conceptionsStatesOldIPUMS}
\end{subfigure}
\end{center}	
\floatfoot{\textsc{Note to figure \ref{fig:conceptionsStatesIPUMS}}: 
Sample consists of all first births from 2005-2013 to white, non-hispanic 25-45 year old 
mothers occurring in the states of Arizona (1,405 births) or Wisconsin (1980 births).}
\end{figure}
%%
%%\begin{figure}[htpb!]
%%  \begin{center}
%%  \caption{Prematurity by Mother's Age}
%%  \label{fig:ARTuse}
%%  \includegraphics[scale=0.8]{./../results/nvss/graphs/prematureAges.eps}
%%  \end{center}
%%\end{figure}
%%

\begin{figure}[htpb!] 
\begin{center}
  \centering  
  \caption{Difference in Good Season by Occupation (Our Definition)}
  \includegraphics[scale=0.72]{./../results/ipums/graphs/birthsOccupation.eps}  
  \label{fig:goodByOcc} 
\end{center}
\vspace{-5mm}
\floatfoot{\textsc{Notes to figure \ref{fig:goodByOcc}}: Groups are defined based on
Sonia's email from December 5. It says: \textbf{Group 1}: occupations for which season 
should not matter (Personal care service 4300-4650; Sales Related 4700-4965; Office-Admin 
support 5000-5940; Food preparation and services 4000-4150), \textbf{Group 2}: demanding 
occupations for which quarter 2 and 3 (especially the summer) should be preferred 
(Practitioners, technicians 3000-3540; Legal 2100-2150; Management 10-430; Life, Physical 
Scientist 1600-1760; Financial Specialists 800-950), \textbf{Group 3}: occupations for 
which season is important and late Spring and Summer are less likely (Education, training, 
library 2200 - 2550).}
\end{figure} 


%\clearpage
\appendix
\section{Appendix Tables}
\input{./../tables/nvssAppendixTables.tex}

\clearpage
%\section{Appendix Figures}
%%\begin{figure}[htpb!] 
%\begin{center}
%  \centering  
%  \caption{Education and Conception Month}
%  \includegraphics[scale=0.7]{./../results/nvss/graphs/conceptionMonthDropout.eps}  
%  \label{fig:concepEduc} 
%\end{center}
%\floatfoot{\textsc{Notes to figure \ref{fig:concepEduc}}: Graph plots the proportion
%of conceptions by month for 25-39 year old women by education level.  Refer to figure
%\ref{bqFig:concepMonth} for additional notes.}
%\end{figure} 

\begin{figure}[htpb!]
\centering
\caption{Younger versus older women births (Spain)}
\label{bqFig:YoungvOldSpain}
  \centering
  \includegraphics[scale=0.82]{../results/spain/graphs/youngMonths.eps}
\floatfoot{\textsc{Notes to figure}: Each point and standard error comes from a regression
of birth month $x$ on a binary indicator of being young (25-39).
}
\end{figure}


\begin{figure}[htpb!]
\begin{center}
\caption{Temperature and Good Quarter (Spain)}
\label{fig:tempSpain}
\begin{subfigure}{.5\textwidth}
  \centering
  \includegraphics[scale=0.55]{./../results/spain/graphs/youngTempCold.eps}
  \caption{Young Mothers}
  \label{fig:tempSpainYoung}
\end{subfigure}%
\begin{subfigure}{.5\textwidth}
  \centering
  \includegraphics[scale=0.55]{./../results/spain/graphs/oldTempCold.eps}
  \caption{Old Mothers}
  \label{fig:tempSpainOld}
\end{subfigure}
\end{center}
\floatfoot{\textsc{Notes to figure}: See notes to figure 
\ref{fig:tempUSAYoung}. Monthly temperature data is 
collected from the Spanish State Meteorology Agency (AEMET).}
\end{figure}



\section{Data Appendix}
\label{bqScn:datApp}
\subsection{US Birth Data}
A brief description of US birth certificate data is provided in section
\ref{bqSscn:USAdata} of the paper.  As discussed, the format of US birth
certificates has undergone two important revisions: The first in 1989
and the second in 2003.  The date of adoption of these revisions varies
by state.  By 2013, 41 states or territories had adopted the revised (2003)
format, while the reaminder still follow the 1989 format.\footnote{The full
birth certificate for each revision is reproduced as figures 1 and 2 in
\citet{MenackerMartin2005}.  Over time the adoption of the 2003 certificate
was as follows: 2005: 12 (31\%), 2006: 19 (49\%), 2007: 22 (53\%), 2008: 27
(65\%), 2009: 28 (66\%), 2010: 33 (76\%), 2011: 36 (83\%), 2012: 38 (86\%),
and 2013: 41 (90\%).  In each case the first number refers to the number
of states, while the parenthesis indicates the percent of births in revised
states.}

In all cases where variable coding differs between the revised and unrevised
certificates (principally education for mother and father), we use the revised
2003 coding of the variables.  The reason we do this is because after 2008,
variables which are exclusive to 1989 certificates are no longer reported.
Figure \ref{bqFig:educMissing} illustrates this pattern.  The dotted line
represents the proportion of observations for maternal education which are
reported in the 1989 format, while the bars represent the proportion
\emph{missing} in the 2003 format.  From 2005-2008, all missing 2003 revision
variables are recorded in the 1989 format.  However, from 2009 onwards only
the 2003 revision of education is reported, meaning that those states who
still use the 1989 standard certificate do not have publicly released education
data.  %As expected, these are not missing at random, given that educational
%attainment varies considerably by state (see table \ref{bqTab:missingEduc}).
%In each case when these variables are used, we include full year and state
%fixed effects.




\begin{figure}[htpb!]
\caption{Missing Education Data by Time}
\label{bqFig:educMissing}
\includegraphics[scale=0.74]{../results/nvss/graphs/missingEduc.eps}
\end{figure}
%\input{./../results/nvss/regressions/NVSSMissingScale.tex}


\subsection{Spanish Data}

Birth certificate records from Spain are released by the National Institute of Statistics (INE)
with coverage from 1979 to 2013 inclusive. These consist of the universe of
births registered annually in Spain. Our principal estimation sample consists of
all first born children who survived one day, born to Spanish mothers. We use
births from the period 2007 to 2013, given that prior to 2007, education was not
recorded on birth certificates.  This results in a sample of 1,239,749 live
births, of which 1,238,685 were singletons.

Like birth certificate data in the US, Spanish certificates provide mother
and child characteristics, including education and labour market status of the
mother (and father where present), mother's age at time of birth, marital
status, and child APGAR, gestation, birth weight, prematurity, and so forth
\citep{INE2013}.  The Spanish records include publicly released data on
geographical location of birth, at both the provincial and municipal level
(similar to US states and counties respectively).

Descriptive statistics for Spanish births are provided in table \ref{bqTab:SumStatsSpain}.  In the same
age group, the average age and proportion of young mothers is similar to data
from USA (32 years and 96\% respectively), however a lower proportion report
being married (64\%), or having at least some post secondary education (53\%).
Spanish newborns are slightly lighter on average than their USA-born
counterparts (3,200g), however are also less likely to be born prematurely,
or classified as having low birth weight.

Spanish climate data at the level of the province is calculated from data released by the State Meteorological
Agency (AEMET). These data record the temperature at principal state
meteorological stations, from which we calculate monthly average, minima and
maxima.



%%\clearpage
%%
%%\section{Tables and Figures Using IPUMS Data (Married Only)}
%%\setcounter{figure}{0}  
%%\renewcommand\thefigure{B.\arabic{figure}}
%%\setcounter{table}{0}
%%\renewcommand\thetable{B.\arabic{table}}
%%\input{./../tables/tablesipumsmarried.tex}
%%\input{./figuresIPUMSmarried.tex}
%%
%%\clearpage
%%
%%\section{Tables and Figures Using IPUMS Data (No Weighting)}
%%\setcounter{figure}{0}  
%%\renewcommand\thefigure{C.\arabic{figure}}
%%\setcounter{table}{0}
%%\renewcommand\thetable{C.\arabic{table}}
%%\input{./../tables/tablesipumsunweight.tex}
%%\input{./figuresIPUMSunweight.tex}
%%


%%NOTE: I TOOK OUT THESE APPENDIX TABLES ON NOV 13 2015.
%%%%\section{Tables Using Conception + 9 Months}
%%%%\setcounter{figure}{0}  
%%%%\renewcommand\thefigure{A.\arabic{figure}}
%%%%\setcounter{table}{0}
%%%%\renewcommand\thetable{A.\arabic{table}}
%%%
%%%%\input{./../tables/tablesexpect.tex}
%%%%\input{./../tables/expectAppendixTables.tex}
%%%
%%%
%%%
%%%%\section{Tables Only For Legally Married Women}
%%%%\renewcommand\thefigure{B.\arabic{figure}}
%%%%\setcounter{figure}{0}  
%%%%\renewcommand\thetable{B.\arabic{table}}
%%%%\setcounter{table}{0}  
%%%%\input{./../tables/tablesmarried.tex}
%%%%\input{./../tables/marriedAppendixTables.tex}


\end{doublespace}
%\end{spacing}
\end{document}







\begin{eqnarray}
\label{struc1}
\max_{\{s,a\}} u(w,b) &\text{s.t.}& w=f(age,educ,\mathbf{X},\varepsilon_w) \\
\label{struc2}
&& b=\mu(age,s,\mathbf{X},\varepsilon_b)                                  \\
\label{struc3}
&& s=\phi(age,educ,\varepsilon_s).
\end{eqnarray}

Here $u(\cdot), f(\cdot), \mu(\cdot)$ and $\phi(\cdot)$ are functions and
$\mathbf{\varepsilon}=(\varepsilon_w,\varepsilon_s,\varepsilon_b)$ are
unobserved shocks.  To close the structural model we must attach functional
form to the four functions, and make distributional assumptions for $\varepsilon$.

Functional form for equations (\ref{struc2}) and (\ref{struc3}) can be quite simple,
as it is precisely what we are estimating in equations (\ref{eqn:season}) and
(\ref{eqn:quality}) respectively.  From our results we know very clearly that:
\[
\frac{\partial b}{\partial s^{good}}>0 \text{\ and \ }
\frac{\partial s^{good}}{\partial age}<0.
\]
It is also generally the case that:
\[
\frac{\partial w}{\partial age}>0.
\]
The wage equation in (\ref{struc1}) could be Mincerian, depending on $age$ and
$age^2$, as well as education, and is calculated at the level of the state,
(perhaps using data from CPS or similar to get wage).  Finally, we define $u(w,b)$
as a Cobb-Douglas
or CES function (or some other logical form), and assume that $\varepsilon$ is
drawn from a trivariate normal.  We can thus write down a log-likelihood function,
and use our birth data to maximise this function, estimating the parameters in
$f(\cdot), \mu(\cdot)$ and $\phi(\cdot)$, which allows us to quantify the
trade-off between ``having a child today'' versus ``having a child tomorrow''
with the relative (market and shadow) prices of labor, as measured by wages, and
child quality, proxied by birthweights.
