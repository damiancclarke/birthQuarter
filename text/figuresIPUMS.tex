\begin{figure}[htpb!]
\centering
\caption{Differences in Prevalence of Good Season Of Birth}
\label{bqFig:YoungvOldIPUMS}
  \centering
  \includegraphics[scale=0.82]{./../results/ipums/graphs/youngQuarter.eps}
\floatfoot{\textsc{Notes to figure}: Each point and standard error comes from a regression
of conception month $x$ on a binary indicator of being young (28-31), versus older (40-45).
}
\end{figure}


\begin{figure}[htpb!] 
\begin{center}
  \centering  
  \caption{Difference in Good Season by Age Group}
  \includegraphics[scale=0.72]{./../results/ipums/graphs/goodSeasonAge.eps}  
  \label{fig:goodByAgeIPUMS} 
\end{center}
\vspace{-5mm}
\floatfoot{\textsc{Notes to figure \ref{fig:goodByAge}}: Coefficients and
standard errors are estimated by regressing ``good season'' on dummies
of maternal age.  Age groups 40-45 are omitted as the base group.  The full
sample consists of mothers aged 20-45.  For the omitted group, proportion 
good season (and standard error) is 0.506(0.008).}
\end{figure} 



\begin{figure}[htpb!]
\begin{center}
\caption{Age and Birth Quarter}
\label{bqFig:birthQuarterIPUMS}
%%\begin{subfigure}{.5\textwidth}
%%  \centering
  \includegraphics[scale=0.72]{./../results/ipums/graphs/birthQuarterAges.eps}
%%  \caption{Proportion of Conceptions in Each Month}
%%  \label{fig:concepAbs}
%%\end{subfigure}%
%%\begin{subfigure}{.5\textwidth}
%%  \centering
%%  \includegraphics[scale=0.55]{./../results/nvss/graphs/conceptionMonthART.eps}
%%  \caption{Proportion of Conceptions (ART Only)}
%%  \label{fig:concepAbsART}
%%\end{subfigure}
\end{center}
\floatfoot{\textsc{Notes to figure \ref{bqFig:birthQuarterIPUMS}}: Birth quarter 
is reported in ACS data. Each line presents the proportion of all births occurring
in each quarter for the relevant age group.}
\end{figure}


\begin{figure}[htpb!]
\begin{center}
\caption{Education and Birth Quarter}
\label{bqFig:concepEducIPUMS}
 \begin{subfigure}{.5\textwidth}
   \centering
   \includegraphics[scale=0.55]{./../results/ipums/graphs/birthQuarterEducYoung.eps}
   \caption{28-31 Year Olds}
   \label{fig:educYoungIPUMS}
 \end{subfigure}%
 \begin{subfigure}{.5\textwidth}
   \centering
   \includegraphics[scale=0.55]{./../results/ipums/graphs/birthQuarterEducOld.eps}
   \caption{40-45 Year Olds}
   \label{fig:educOldIPUMS}
 \end{subfigure}
 \end{center}
 \floatfoot{\textsc{Notes to figure \ref{bqFig:concepEducIPUMS}}: Each line 
presents the proportion of all births conceived in each month for the relevant 
age group and education level. Refer to figure \ref{bqFig:concepMonth} 
for additional notes.}
\end{figure}

 
 \begin{figure}[htpb!]
 \begin{center}
   \centering
   \caption{Good Season by State (Young)}
   \includegraphics[scale=0.7]{./../results/ipums/graphs/youngGoodSeason.eps}
   \label{fig:mapYoungIPUMS}
 \end{center}
 \end{figure}
 
 \begin{figure}[htpb!]
 \begin{center}x
   \centering
   \caption{Good Season by State (Old)}
   \includegraphics[scale=0.7]{./../results/ipums/graphs/oldGoodSeason.eps}
   \label{fig:mapOldIPUMS}
 \end{center}
 \floatfoot{\textsc{Notes to figures \ref{fig:mapYoungIPUMS}-\ref{fig:mapOldIPUMS}}: 
 State data consists of all first births to white non-hispanic women from 2005 
 and 2013.  Figure \ref{fig:mapYoungIPUMS} includes all mothers aged 28-31, while
 figure \ref{fig:mapOldIPUMS} includes mothers aged 40-45.}
 \end{figure}
 
 
 \begin{figure}[htpb!]
 \begin{center}
 \caption{Temperature and Good Quarter}
 \label{fig:tempUSAIPUMS}
 \begin{subfigure}{.5\textwidth}
   \centering
   \includegraphics[scale=0.55]{./../results/ipums/graphs/youngTempCold.eps}
   \caption{Young Mothers (28-31)}
   \label{fig:tempUSAYoungIPUMS}
 \end{subfigure}%
 \begin{subfigure}{.5\textwidth}
   \centering
  \includegraphics[scale=0.55]{./../results/ipums/graphs/oldTempCold.eps}
  \caption{Old Mothers (40-45)}
  \label{fig:tempUSAOldIPUMS}
\end{subfigure}
\end{center}
\floatfoot{\textsc{Notes to figure \ref{fig:tempUSAIPUMS}}: Each point 
represents a state average of the proportion of women giving birth in the 
good birth season between 2005 and 2014.  The dotted line is a fitted 
regression line.  Monthly temperature data is collected from the National 
Centers for Environmental Information.  States with less than 500 births
in ACS over the entire period of analysis are not displayed.}
\end{figure}

\begin{figure}[htpb!]
  \begin{center}
  \caption{Difference in Births (\% Good Season - \% Bad Season)}
  \label{fig:NVSSbirthsAgesIPUMS}
  \includegraphics[scale=0.8]{./../results/ipums/graphs/birthQdiff_4Ages.eps}
  \end{center}
\end{figure}
%%
%%\begin{figure}[htpb!] 
%%\begin{center}
%%  \centering  
%%  \caption{ART Conceptions by Month}
%%  \includegraphics[scale=0.7]{./../results/nvss/graphs/proportionMonthART.eps}  
%%  \label{fig:ARTMonth} 
%%\end{center}
%%\floatfoot{\textsc{Notes to figure \ref{fig:ARTMonth}}: Proportion of ART births
%%are calculated using data from 2012-2013 for our main sample.  The proportion
%%is calculated as: (ART conceptions)/(Non-ART Conceptions + ART Conceptions).}
%%\end{figure} 
%%
%%\begin{figure}[htpb!]
%%\begin{center}
%%\caption{Difference in Births (\% Good Season - \% Bad Season)}
%%\label{fig:birthDiffART}
%%\begin{subfigure}{.5\textwidth}
%%  \centering
%%  \includegraphics[scale=0.55]{./../results/nvss/graphs/birthQdiff_4AgesART.eps}
%%  \caption{ART Usage}
%%  \label{fig:DiffNoART}
%%\end{subfigure}%
%%\begin{subfigure}{.5\textwidth}
%%  \centering
%%  \includegraphics[scale=0.55]{./../results/nvss/graphs/birthQdiff_4AgesNoART.eps}
%%  \caption{No ART Users}
%%  \label{fig:DiffART}
%%\end{subfigure}
%%\end{center}
%%\end{figure}
%%
\begin{figure}[htpb!]
\begin{center}
\caption{Conceptions by Month (Arizona and Wisconsin)}
\label{fig:conceptionsStatesIPUMS}
\begin{subfigure}{.5\textwidth}
  \centering
  \includegraphics[scale=0.55]{./../results/ipums/graphs/birthQuarterArizonaWisconsin_young.eps}
  \caption{Young Women (28-31)}
  \label{fig:conceptionsStatesYoungIPUMS}
\end{subfigure}%
\begin{subfigure}{.5\textwidth}
  \centering
  \includegraphics[scale=0.55]{./../results/ipums/graphs/birthQuarterArizonaWisconsin_old.eps}
  \caption{Old Women (40-45)}
  \label{fig:conceptionsStatesOldIPUMS}
\end{subfigure}
\end{center}	
\floatfoot{\textsc{Note to figure \ref{fig:conceptionsStatesIPUMS}}: 
Sample consists of all first births from 2005-2013 to white, non-hispanic 25-45 year old 
mothers occurring in the states of Arizona (1,405 births) or Wisconsin (1980 births).}
\end{figure}
%%
%%\begin{figure}[htpb!]
%%  \begin{center}
%%  \caption{Prematurity by Mother's Age}
%%  \label{fig:ARTuse}
%%  \includegraphics[scale=0.8]{./../results/nvss/graphs/prematureAges.eps}
%%  \end{center}
%%\end{figure}
%%

\begin{figure}[htpb!] 
\begin{center}
  \centering  
  \caption{Difference in Good Season by Occupation (Our Definition)}
  \includegraphics[scale=0.72]{./../results/ipums/graphs/birthsOccupation.eps}  
  \label{fig:goodByOcc} 
\end{center}
\vspace{-5mm}
\floatfoot{\textsc{Notes to figure \ref{fig:goodByOcc}}: Groups are defined based on
Sonia's email from December 5. It says: \textbf{Group 1}: occupations for which season 
should not matter (Personal care service 4300-4650; Sales Related 4700-4965; Office-Admin 
support 5000-5940; Food preparation and services 4000-4150), \textbf{Group 2}: demanding 
occupations for which quarter 2 and 3 (especially the summer) should be preferred 
(Practitioners, technicians 3000-3540; Legal 2100-2150; Management 10-430; Life, Physical 
Scientist 1600-1760; Financial Specialists 800-950), \textbf{Group 3}: occupations for 
which season is important and late Spring and Summer are less likely (Education, training, 
library 2200 - 2550).}
\end{figure} 
